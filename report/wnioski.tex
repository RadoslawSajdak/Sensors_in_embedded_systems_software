\chapter{Wnioski}
\label{cha:results}
W trakcie trwania projektu poruszono zarówno aspekty dotyczące projektowania sprzętu, jak i tworzenia oprogramowania. Sam projekt, na pewno był projektem złożonym i wymagającym. Zaprojektowanie płytki, wymagało czasochłonnej analizy dokumentacji układów, a nawet przypomnienia sobie zagadnień elektroniki analogowej. Mimo to, popełniono wiele błędów opisanych we wcześniejszych rozdziałach. W trakcie tworzenia kolejnej rewizji sprzętu, należałoby więc uwzględnić omówione zmiany. Dodatkowo, elementy pasywne w obudowach 0805, które wydawały się dobrą decyzją pod kątem ręcznego lutowania płytek, okazały się niepotrzebnym utrudnieniem. W kolejnej rewizji, należałoby zastosować elementy w obudowach 0603. Dzięki temu, oszczędzonoby wiele miejsca, co pozwoliłoby na efektywniejsze wykorzystanie dostępnej powierzchnii. Kolejnym istotnym wnioskiem, jest sam dobór czujników na płytce. W przypadku tego projektu, głównym założeniem było przetestowanie różnych interfejsów na płytce z mikroprocesorem. W przypadku płytki komercyjnej, zastosowanie czujnika z grzałką obok czujnika temperatury, byłoby zupełnie pozbawione sensu. Wyjątkiem, byłoby dodanie układu sterującego zasilaniem czujnika z grzałką. Cennym wnioskiem wydaje się być również fakt, że błędy w przypadku projektowania sprzętu, są nieuniknione i w przypadku chęci stworzenia produktu, na etapie planowania należy uwzględnić wykonanie kilku kolejnych rewizji sprzętu. W przypadku tego projektu, mimo sprawdzania layoutu oraz schematu przez kilka osób z doświadczeniem, błędy i tak się pojawiły. W przypadku tworzenia oprogramowania, najważniejszym z wniosków wydaje się być fakt, że należy wzajemnie sprawdzać kod w obrębie zespołu. Pozwala to nie tylko wychwycić wiele błędów, ale przede wszystkim nauczyć się zupełnie nowych sposobów tworzenia kodu. Błędem, była natomiast próba stworzenia oprogramowania działającego na dwóch zupełnie różnych mikroprocesorach. Decyzja o takim podejściu podyktowana była założeniem, że HAL dostarczony przez ST, rozwiązuje kwestie kompatybilności między układami. Tymczasem okazało się, że wyjątkowo dużo czasu zostało zmarnowane na próby uruchomienia kodu stworzonego na innej płytki. Pojawiające się rozbieżności w nazwach stałych z bibliotek ST, różne sposoby konfiguracji peryferiów oraz potrzeba korzystania z różnych bibliotek sprawiła, że część kodu była tak na prawdę pisana dwa razy.\newline
W tym wszystkim należy zaznaczyć, że stworzony projekt znacząco przekraczał podstawowe założenia przedmiotu, jednak od początku jego celem była chęć praktycznego sprawdzania omawianych zagadnień. Można więc powiedzieć, że jego wykonanie było sztuką dla sztuki. Dzięki temu, pozwolił on nie tylko na zdobycie nowych umiejętności, ale również przyjemne spędzenie czasu.
