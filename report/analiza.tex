\chapter{Analiza wyników}
\label{cha:analysis}


\section{Wyznaczenie rezystancji wewnętrznej sieci}
\label{sec:resistance_in}
Pierwszym z zadań, było wyznaczenie rezystancji wewnętrznej sieci. Wykorzystano do tego dane, zebrane w trakcie innych pomiarów.
\newline Na początek, wyznaczono napięcie sieci przy podłączonym obciążeniu. Według normy PN-IEC60038, napięcie sieci powinno wynosić 230V oraz mieć częstotliwość 50Hz. Przebieg \ref{img:input_load_50_sch} przesstawia napięcie na wejściu układu. Widać, że nie jest to idealny sinus. Wartość skuteczna tego przebiegu, została obliczona na 233.7V, co mieści się w zakładanych przez normę $230V\pm 10\%$. Rezystancję odbiornika obliczono jako:
\newline

\begin{figure}[H]
    \centering
    \includegraphics[width=12cm]{Graphics/input_50w_load.png}
    \caption{Przebieg napięcia układu obciążonego żarówkami o mocy 50W}
    \label{img:input_load_50_sch}
\end{figure}
$ R_{ob} = \frac{U_{wy}}{I_{wy}} = \frac{285.96}{0.154} =  1856.88 \Omega$
\newline Następnie, obliczono wartość modułu impedancji: \newline
$ Z_m = R_{ob}(\frac{E}{U} - 1) = 1856.88 * (\frac{233.9}{233.7} - 1) = 1.59\Omega $
\newline Na podstawie przebiegu \ref{img:nieustalone100W}, ustalono wartość prądu zwarcia na 1.97A. W sekcji \ref{sec:voltage_not_connected} wyznaczono napięcie skuteczne badanej sieci na równe 233.9V. Dzięki powyższym danym, wyznaczono rezystancję sieci:\newline
$R_w = \frac{E - U}{I_{zw}} = \frac{233.9 - 233.7}{1.97} = 162m\Omega $
\newline Jest to wartość tak mała, że nie wpływała ona w sposób znaczący na dokonywane w trakcie ćwiczenia pomiary.


\section{Obserwacja przebiegów napięcia sieciowego}
\label{sec:voltage_not_connected}
W zadaniu, dokonano pomiaru napięcia sieciowego oraz napięcia po transformatorze separujacym PFS800. Rysunek \ref{img:empty_measurement} przedstawia zarejestrowany przez oscyloskop. Ponieważ skrypt, mający zapisywać dane, nie działał poprawnie przy uruchomionym FFT, wykonano ją na surowych danych przy użyciu skryptu w języku python. Otrzymane przebiegi przedstawia rysunek \ref{img:emmpty_fft}. Można zauważyć, że w obu przypadkach, przebieg nie jest idealnym przebiegiem sinusoidalnym. Kolorem pomarańczowym, na przebiegu zaznaczono funkcję $ 330 * sin(2\pi * 50 * t + \frac{\pi}{2}) $. Widoczna różnica w okolicy wartości szczytowych, jest wynikiem wpływu kolejnych harmonicznych na przebieg napięcia sieci. Warto dodać, że w uzyskanym FFT, pierwszą z wyraźnie widocznych harmonicznych jest piąta harmoniczna, co najprawdopodobniej wynika ze zbyt małej ilości danych.\newline
Wartość skuteczna napięcia badanego przebiegu, wynosi 233.9V 


\begin{figure}[H]
    \centering
    \includegraphics[width=12cm]{Graphics/siec_measure_pusta.PNG}
    \caption{Przebieg napięcia sieciowego}
    \label{img:empty_measurement}
\end{figure}

\begin{figure}[H]
    \centering
    \includegraphics[width=12cm]{Graphics/empty_fft.png}
    \caption{Przebieg napięcia sieciowego oraz FFT otrzymane skryptowo}
    \label{img:emmpty_fft}
\end{figure}

Opisany wyżej pomiar, powtórzono stosując transformator separujący PFS800. Przebiegi zaobserwowane na oscyloskopie, różnią się wyraźnie. Rysunek \ref{img:trafo_measurement} przedstawia zrzut przebiegu z oscyloskopu. Wyraźnie widać na nim, że kształtem bardziej przypomina funkcję sinus. Należy jednak zauważyć, że maksymalne napięcie oraz napięcie skuteczne, wynoszą kolejno 346V i 245V. Wartości te mieszczą się w opisywanej wcześniej normie. Ponieważ producent deklaruje napięcie skuteczne 230 zgodne z obowiązującą normą, rozbieżność wynika prawdopodobnie z przekładni nieznacznie różnej od jedności.

\begin{figure}[H]
    \centering
    \includegraphics[width=12cm]{Graphics/siec_trafo.PNG}
    \caption{Przebieg napięcia po transformatorze separujacym}
    \label{img:trafo_measurement}
\end{figure}

\begin{figure}[H]
    \centering
    \includegraphics[width=12cm]{Graphics/trafo_fft.PNG}
    \caption{Przebieg napięcia sieciowego po transformatorze separującym oraz FFT otrzymane skryptowo}
    \label{img:trafo_fft}
\end{figure}
%%%%%%%%%%%%%%%%%%%%%%%%%%%%%%%%%%%%%%%%%%%%%%%%%%%%%%%%%%%%%
\section{Pomiary prostownika z mostkiem Graetza}
\label{sec:Graetz}
W kolejnym kroku, dokonano pomiarów:
\begin{itemize}
    \item Napięcia i prądu sieci
    \item Napięcia i prądu na wyjściu obciążenia
    \item Napięcia i prądu udarowego podzas włączania i wyłączania układu
\end{itemize}
Jako obciążenie, zastosowano opisany w podrozdziale \ref{sub:bulbs} model z żarówkami żarowymi.
\newline
Znaczącą różnicę widać, porównując przebiegi z rysunku \ref{img:graetz50w} do \ref{img:graetz200w}. Dużo większy prąd, pobierany przez układ, spowodował znaczący wzrost amplitudy tętnień w na wyjściu układu.
Kanały reprezentują kolejno:
\begin{itemize}
    \item Kanał 1 - Napięcie na wejściu układu
    \item Kanał 2 - Prąd na wejściu układu
    \item Kanał 3 - Prąd na wyjściu mostka
    \item Kanał 4 - Napięcie na wyjściu mostka
\end{itemize}

\begin{figure}[H]
    \centering
    \includegraphics[width=12cm]{Graphics/prostownik_25w.PNG}
    \caption{Układ z mostkiem Graetza - 50W na wyjściu układu}
    \label{img:graetz50w}
\end{figure}

\begin{figure}[H]
    \centering
    \includegraphics[width=12cm]{Graphics/prostownik100w.PNG}
    \caption{Układ z mostkiem Graetza - 200W na wyjściu układu}
    \label{img:graetz200w}
\end{figure}

Dla porównania, przeprowadzono symulację badanego układu w programie LTspice. Odwzorowano badany schemat, przedstawiony na rysunku \ref{img:graetz_sch}. Jako źródło, użyto źródła napięcia sinusoidalnego, o parametrach wyznaczonych w sekcji \ref{sec:resistance_in}. Rysunek \ref{img:graetz_ltspice_sch} przedstawia schemat stworzonego układu.\newline

\begin{figure}[H]
    \centering
    \includegraphics[width=15cm]{ltspice/graetz_ltspice_sch.png}
    \caption{Układ z mostkiem Graetza - schemat programu ltspice}
    \label{img:graetz_ltspice_sch}
\end{figure}

Wyniki przeprowadzonej analizy czasowej, widoczne są na rysunku \ref{img:graetz_ltspice_calc}. Można zauważyć, że symulacja w przybliżeniu pokazała spodziewane przebiegi. Niestety, ze względu na dużą ilość obliczeń, nie są one tak dokładne jak obserwowane w sieci. Mniejszy krok analizy czasowej, powodował niewydolność oprogramowania LTspice. \newline

\begin{figure}[H]
    \centering
    \includegraphics[width=15cm]{ltspice/graetz_ltspice_calc.png}
    \caption{Układ z mostkiem Graetza - przebiegi czasowe sygnałów}
    \label{img:graetz_ltspice_calc}
\end{figure}

Następnie, zbadano prądy udarowe układu z mostkiem Graetza. Kolejność połączeń na wejściach oscyloskopu, nie zmieniła się. Zaobeserwowane przebiegi przedstawia rysunek \ref{img:nieustalone100W}
Można na nich zaobserwować znaczące skoki prąu na wejściu oraz wyjściu układu. Nie zależą one jednak od obciążenia, a od pojemności znajdujących się w układzie elementów. Skok do nawet 2A na wejściu układu pokazuje, jak niebezpieczne dla układów mogą być stany nieustalone, powstające podczas włączania i wyłączania układu.

\begin{figure}[H]
    \centering
    \includegraphics[width=12cm]{Graphics/nieustalone100w.PNG}
    \caption{Układ z mostkiem Graetza - Stany nieustalone}
    \label{img:nieustalone100W}
\end{figure}
%%%%%%%%%%%%%%%%%%%%%%%%%%%%%%%%%%%%%%%%%%%%%%%%%%%%%%%%
\section{Pomiary prostownika z mostkiem Valley-fill}
Pomiary wykonane w sekcji \ref{sec:Graetz} wykonano ponownie dla prostownika z mostkiem Valley-fill. Schemat badanego układu przdstawiono na rysunku \ref{img:valley_sch}. Na rysunku \ref{img:valley40W} przedstawiono przebiegi dla obciążenia 40W. Warte uwagi, okazało się FFT badanego układu, przedstawione na rysunku \ref{img:valley40W_fft}. Widać, że badany sygnał ma bardzo dużo harmonicznych. Przebiegi z obciążeniem 100W, przedstawiono na rysunku \ref{img:valley100W}. Na przebiegu kanału 4, widać pojawiające się niedoskonałości napięcia wyjściowego. Dużo lepiej jednak, pokazuje to rysunek \ref{img:valley100W_fft}. Można zauważyć, że wraz ze wzrostem obciążenia, przebieg wyjściowy przestaje przypominać sinus, wyprostowany dwupołówkowo. Znacząco rośnie też ilość harmonicznych w sygnale.

\begin{figure}[H]
    \centering
    \includegraphics[width=12cm]{Graphics/valley40w.PNG}
    \caption{Układ z mostkiem Valley-fill - Obciążenie 40W}
    \label{img:valley40W}
\end{figure}

\begin{figure}[H]
    \centering
    \includegraphics[width=12cm]{Graphics/valley40_fft.PNG}
    \caption{Układ z mostkiem Valley-fill FFT - Obciążenie 40W}
    \label{img:valley40W_fft}
\end{figure}

\begin{figure}[H]
    \centering
    \includegraphics[width=12cm]{Graphics/valley100w.PNG}
    \caption{Układ z mostkiem Valley-fill - Obciążenie 100W}
    \label{img:valley100W}
\end{figure}

\begin{figure}[H]
    \centering
    \includegraphics[width=12cm]{Graphics/valley100_fft.PNG}
    \caption{Układ z mostkiem Valley-fill FFT - Obciążenie 100W}
    \label{img:valley100W_fft}
\end{figure}

Również dla tego mostka, przeprowadzono wstępne symulacje w programie LTspice. Schemat układu przedstawia rysunek \ref{img:valley_ltspice_sch}. W przypadku tego układu, symulacja była wyjątkowo czasochłonna, ze względu na dużą złożoność względem mostka Graetza. Przebiegi symulacji z rysunku \ref{img:valley_ltspice_calc}, odbiegają od otrzymanych wyników. Mimo to, symulacja w wystarczająco dobrym stopniu pokazała, jakiego rzędu wartości oraz jakich (w przybliżeniu) przebiegów, możgliśmy się spodziewać.

\begin{figure}[H]
    \centering
    \includegraphics[width=15cm]{ltspice/valley_ltspice_sch.png}
    \caption{Układ z mostkiem Valley-fill - schemat programu ltspice}
    \label{img:valley_ltspice_sch}
\end{figure}

\begin{figure}[H]
    \centering
    \includegraphics[width=15cm]{ltspice/valley_ltspice_calc.png}
    \caption{Układ z mostkiem Valley-fill - symulacja programu ltspice}
    \label{img:valley_ltspice_calc}
\end{figure}

Kolejnym krokiem, była obserwacja stanów nieustalonych występujących w układzie z mostkiem Valley-fill. Na rysunku \ref{img:valley100W_nieustalone} wytaźnie widać, co dzieje się w badanym układzie w momencie włączania. Podczas uruchomienia, przez układ płyną prądy o nawet kilkukrotnie większej wartości, niż w stanie ustalonym. Zaobserwowany na wyjściu układu skok, miał wartość aż 1.41A, podczas gdy w stanie ustalonym, wartość ta nie przekracza 215mA. Podobnie jak w przypadku układów z mostkiem Graetza, ćwiczenie pokazuje jak niebezpieczne mogą okazać się momenty włączania układu.

\begin{figure}[H]
    \centering
    \includegraphics[width=12cm]{Graphics/vnieustalone100w.PNG}
    \caption{Układ z mostkiem Valley-fill - Prądy i napięcia udarowe}
    \label{img:valley100W_nieustalone}
\end{figure}
%%%%%%%%%%%%%%%%%%%%%%%%%%%%%%%%%%%%%%%%%%%%%%%%%

\section{Pomiary zaburzeń układu aktywnego korektora mocy PFC}
Ze względu na zwarcie w sieci, nie dokonano pomiarów aktywnego korektora mocy. Podczas pomiarów, spodziewano się zaobserwować przesunięcie fazowe między prądem i napięciem bliskie zera. Również przebieg sygnałów według oczekiwań, miałby się pokrywać. Dzięki temu, można uzyskać bardzo dobrą wartość współczynnika mocy, który jak wspomniano w sekcji \ref{sec:theoretical_introduction}, zależy właśnie od przesunięcia fazowego.
